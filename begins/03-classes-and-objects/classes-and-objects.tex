\documentclass{beamer}

% Copyright 2010 Drow Ltd.
% 
% In principle, this file can be redistributed and/or modified under
% the terms of the GNU Public License, version 2.
% 
% However, this file is supposed to be a template to be modified
% for your own needs. For this reason, if you use this file as a
% template and not specifically distribute it as part of a another
% package/program, I grant the extra permission to freely copy and
% modify this file as you see fit and even to delete this copyright
% notice. 
\mode<presentation>
{
  \usetheme[titleline=true,
  alternativetitlepage=true,
  titlepagelogo=images/Java_logo]{Torino}
  \usecolortheme{nouvelle}
  \beamertemplatenavigationsymbolsempty
}

\usepackage{times}
\usepackage[utf8]{inputenc}
\usepackage[english,bulgarian]{babel}
\usepackage[T2A]{fontenc}

\usepackage{listings}
\lstset{language=Java,
  captionpos=b,
  tabsize=4,
  keywordstyle=\color{blue},
  commentstyle=\color{gray},
  stringstyle=\color{green},
  numbers=left,
  breaklines=true,
  showstringspaces=false,
  basicstyle=\ttfamily,
  emph={label},
  frame=shadowbox, 
  rulesepcolor=\color{blue},
  columns=fixed}

\title{Класове и обекти в Java}

\author{инж. Божидар ~Бацов}

\institute{Drow Ltd.}

\date{9.11.2010}

\subject{Talks}
% This is only inserted into the PDF information catalog. Can be left
% out. 

\begin{document}

\begin{frame}
  \titlepage
\end{frame}

\begin{frame}{Съдържание}
  \tableofcontents[pausesections]
\end{frame}


\section{Обектно ориентирано програмиране}

\subsection{Какво е ООП?}

\begin{frame}{Обектно ориентирано програмиране}
  \transdissolve
  \begin{itemize}
  \item Какво е ООП?
  \item Основния понятия в ООП
    \begin{itemize}
      \item клас - шаблон
      \item обект - инстанция на шаблон
        \begin{itemize}
          \item състояние(state)
          \item поведение(behaviour)
        \end{itemize}
    \end{itemize}
  \end{itemize}
\end{frame}

\begin{frame}{Пример - човек}
  \begin{columns}
    \column{.5\textwidth}
    Клас човек
    \begin{itemize}
      \item атрибути
        \begin{itemize}
          \item име
          \item възраст 
          \item адрес
        \end{itemize}
      \item поведение
        \begin{itemize}
          \item говори
          \item ходи
        \end{itemize}
    \end{itemize}

    \column{.5\textwidth}
    Човек Божидар
    \begin{itemize}
    \item състояние
      \begin{itemize}
      \item Божидар
      \item 33 
      \item Готам Сити
      \end{itemize}
    \item поведение
      \begin{itemize}
      \item Кажи "аааа"
      \item Ходи в бат пещерата
      \end{itemize}
    \end{itemize}
  \end{columns}
\end{frame}

\begin{frame}{Основни понятия при класовете}
\transdissolve
\begin{itemize}
\item Атрибути - реализирано чрез полета(instance fields)
\item Поведение - реализирано чрез методи(instance methods)
\item Капсулиране на състоянието - достъпно само чрез методи
\item Класово състояние/поведение - статични полета и методи
\item Наследяване
\end{itemize}
\end{frame}

\begin{frame}{Основни понятия при обектите}
  \transdissolve
  \begin{itemize}
  \item Състояние - стойности за атрибутите
  \item Поведение - операции над състоянието
  \item Идентичност - каква информация определя два обекта като
    еднакви 
  \end{itemize}
\end{frame}

\begin{frame}{Отношения между класовете}
  \transdissolve
  \begin{itemize}
  \item Зависимост(coupling, dependency, uses-a) - класове използвани от класа
  \item Агрегация(has-a) -  индикира че един обект съдържа други обекти
  \item Наследяване(is-a) - индикира че един e  тип подтип(по-специализиран тип) на друг тип
  \end{itemize}
\end{frame}

\begin{frame}{UML}
  \transdissolve
  \begin{itemize}
  \item   Unified modeling language

  \item  Показва зависимости между класовете(и
    много други неща) във вид на диаграми
  \item IntelliJ IDEA и  NetBeans има вградена поддръжка на UML

  \end{itemize}
\end{frame}

\begin{frame}{Създаване на обекти}
  \transdissolve
  \begin{itemize}
  \item   Конструктор – специален метод, който
    инициализира състоянието на обект

   \item  Конструктора има същото име като
    името на класа
    \item Обръщения към конструктора трябва да
    се предшествани от ключовата дума
    \textbf{new}
  \end{itemize}
\end{frame}

\begin{frame}[fragile]
  \frametitle{Създаване на обекти - пример}
  \transdissolve
\begin{lstlisting}
Date currentDate; //not initialized
currentDate = new Date(); //create a new date object
System.out.println(currentDate.toString());
anotherDate = currentDate; // a second referrence to the first date
\end{lstlisting}
\end{frame}

\begin{frame}{Променливи от референтен тип}
  \transdissolve
  \begin{itemize}
  \item   Променливите НЕ са обекти
  \item  Променливите НЕ съдържат обект
  \item  Променливите съдържат само
    препратка(референция, адрес) на обект
  \item  Няколко променливи могат да сочат към
    един обект
  \item  Променливите не са инициализирани с
    null по подразбиране
  \end{itemize}
\end{frame}

\begin{frame}{Класове от стандартната библиотека}
  \transdissolve
  \begin{itemize}
  \item java.util.Date -  представя дата
  \item java.util.Random - генератор на случайни числа
  \item java.util.Scanner - вход от поток
  \item java.lang.System - агрегатор на системни
    операции
  \item java.util.GregorianCalendar -  представя дата по
    грегорианския(нашия) календар

  \end{itemize}
\end{frame}

\begin{frame}{Класът String}
  \transdissolve
  \begin{itemize}
  \item Представя низ от символи, кодирани в
    Unicode-16
    \item Непроменими инстанции
    \item специален опростен синтаксис
    \begin{itemize}
    \item Създаване на обект
    \lstinline$String someString = "string";$
    \item Конкатениране(сливане) на низове    
    \lstinline$String sum = "first " + "second";$
    \end{itemize}

  \end{itemize}
\end{frame}

\begin{frame}{Разработка на класа Person}
  \transdissolve
  \begin{itemize}
  \item Полета - име, фамилия, адрес, дата на раждане
  \item Конструктoри
  \item Методи - аксесори/мутатори, кажи нещо
  \end{itemize}
\end{frame}

\begin{frame}{Работа с няколко сорс файла}
  \transdissolve
  \begin{itemize}
  \item   Java компилатора разбира
    зависимостите между класовете
  \item  Ако клас А зависи(използва) от клас Б –
    командата:
    javac A.java
    ще компилира и класа Б, ако не е вече
    компилиран
   \item  Вградена функционалност тип \textbf{make}
  \end{itemize}
\end{frame}

\begin{frame}{Изследване на класа Person}
  \transdissolve
  \begin{itemize}
  \item    Всички полета са маркирани с
    модификатора за достъп “private”

   \item Private ограничава достъпа до полетата
    само до рамките на класа, в който те са
    дефинирани

   \item Осигурява се отлична капсулация на
    данните и достъпа до тях се контролира
    надеждно чрез методи

  \end{itemize}
\end{frame}

\begin{frame}{Изследване на класа Person}
  \transdissolve
  \begin{itemize}
  \item   Всички методи са маркирани с
    модификатора за достъп public
  \item  Public маркира програмен елемент като
    достъпен за ВСИЧКИ класове

  \item  Почти никога не е добра идея полета да
    бъдат маркирани като public – това
    нарушава капсулацията на данните –
    основен принцип на ООП
  \end{itemize}
\end{frame}

\begin{frame}{Представяне на обект като низ}
  \transdissolve
  \begin{itemize}
  \item Полезна техника за търсене на грешки
  \item Метод toString()
  \item Генериране на метода toString() в IntelliJ IDEA
  \end{itemize}
\end{frame}

\begin{frame}{Конструктор}
  \transdissolve
  \begin{itemize}
  \item има същото име като класа
  \item Не връща резултат
  \item  Клас може да има повече от един
    конструктор
  \item  Конструктора може да приема различен
    брой аргументи
  \item  Конструктора може да бъде извикан
    само с ключовата дума \textbf{new}
  \end{itemize}
\end{frame}

\begin{frame}{Параметри на конструкторите}
  \transdissolve
  \begin{itemize}
  \item Преки/Директни 
    стойността им се
    използва директно за инициализиране
    на поле
  \item Непреки – 
    стойността им се
    използва като база за изчисляването на
    стойността, която да се зададе на поле
    \item Препоръчително е конструкторът да няма повече от 4 параметъра
  \end{itemize}
\end{frame}

\begin{frame}{Предимства на енкапсулацията}
  \transdissolve
  \begin{itemize}
  \item   Валидиране на данни
  \item  Защитаване на данни от промени
  \item  Гъвкавост -  възможно е да бъде
    променена вътрешна имплементация
    без промени на публичното API

  \end{itemize}
\end{frame}

\begin{frame}[fragile]
  \frametitle{Допълнение за private}
  \transdissolve
    private членовете от всеки обект на
    класа са достъпни в неговите методи.
\begin{lstlisting}
  class Person {
    public int compare(Person another) {
        if (this.name.equals(another.name)) ...
    }
\end{lstlisting}
\end{frame}


\begin{frame}{Добрият стил диктува...}
  \transdissolve
  ... Само методите, които ще бъдат
  използвани от клиентски код да бъдат
  декларирани публични –
  допълнителните(utility) методи е
  желателно да бъдат private

\end{frame}

\begin{frame}{Модификатора final}
  \transdissolve
  \begin{itemize}
  \item Приложен към примитивен тип -  има
    смисъл на константа
  \item  Приложен към обект -  има смисъл на
    константна референция, самия обект
    може да бъде променен
  \item  Приложен към клас -  ненаследим клас
  \item  Приложен към метод -  метод, който не
    може да бъде overridden
  \end{itemize}
\end{frame}

\begin{frame}{Статични(класови) полета и методи}
  \transdissolve
  \begin{itemize}
  \item   Асоциирани са съм самия клас, а не
    обектите инстанцирани от него
  \item  Достъпни са както посредством класа,
    така и посредство обектите от него
  \item  Статични полета често се използват за
    съхранение на константи
  \item  Статични методи често се използват за
    реализация на фабрични методи

  \end{itemize}
\end{frame}

\begin{frame}{Параметри на методите}
  \transdissolve
  \begin{itemize}
  \item Винаги се предават по стойност(by value)
    \begin{itemize}
      \item При примитивните типове се предава самата стойност на
        аргумента
      \item При референтните типове се предава препратка(адрес) към
        обекта. Самата препратка е непроменима, но обектът указан от
        нея е.
    \end{itemize}

  \item Прототип на метод - името и параметрите на метода, но не и
    връщаната му стойност
  \end{itemize}
\end{frame}

\begin{frame}{Конструиране на обекти}
  \transdissolve
  \begin{itemize}
  \item Обекти обикновено се създават посредством конструктор
  \item Конструктор по подразбиране
  \item Всички полета се инициализират със стойност по подразбиране
    \begin{itemize}
      \item 0 за числени типове
      \item null за референтни типове
      \item false за булеви полета
    \end{itemize}
  \item Може да имате повече от един конструктор
  \item Може да извиквате един конструктор от друг
  \end{itemize}
\end{frame}

\begin{frame}{Презареждане на методи}
  \transdissolve
  \begin{itemize}
  \item Може да дефинирате няколко метода с едно и също име в даден
    клас стига сигнатурата им да е различна
  \item Компилаторът използва сигнатурата, за да определи правилният
    метод, който да бъде извикан
  \item Сигнатура(прототип)
    \begin{itemize}
      \item формални параметри
      \item име на метода
      \item НЕ включва типа на връщаната от метода стойност
    \end{itemize}
  \end{itemize}
\end{frame}

\begin{frame}{Съвети за работа с конструктори}
  \transdissolve
  \begin{itemize}
  \item Именуване на параметрите
  \item Навързване на конструктори
  \item Конструктори с повече от 4-5 параметъра могат да бъдат
    заменени с шаблона за дизайн Builder
  \item Конструкторите могат да бъдат заменени със статични
    методи-фабрика, които вътрешно използват private конструктори
  \end{itemize}
\end{frame}

\begin{frame}{Пакети в Java}
  \transdissolve
  \begin{itemize}
  \item позволяват групиране на свързани класове
  \item предотвратяват конфликти в имената на класове - приблизителен
    еквивалент на namespaces в C++/Java
  \item съществува строга връзка между имената нивата в имената на
    пакетите съответстват на директории във файловата система
  \item желателно е да са уникални - конвенцията диктува да започват
    обърнат наопаки домейнът на вашата организация
  \item запазени имена на пакети - java.*, javax.*
  \end{itemize}
\end{frame}

\begin{frame}{Импортиране на класове и пакети}
  \transdissolve
  \begin{itemize}
  \item Всеки клас има достъп до всички класове в неговия пакет без
    допълнителни импорти
  \item Всеки клас има достъп до всички публични класове, които са
    импортирани или използвани с напълно квалифицирани имена
  \item Всички класове от един пакет могат да бъдат импортирани с
    wildcard import, но това не е препоръчително
  \end{itemize}
\end{frame}

\begin{frame}{Статични импорти}
  \transdissolve
  \begin{itemize}
  \item Импортиране само на определени статични елементи от даден клас
    - полета или методи
  \item Подходяща техника за работа със статични класове като
    java.util.Math
  \item Могат да нарушат четимостта на програмата, ако се използват безразборно
  \end{itemize}
\end{frame}

\begin{frame}{Особености на пакетите}
  \transdissolve
  \begin{itemize}
  \item Всяка част от името на пакета е отделена от другите с точка и
    съответства на поддиректория в директорията с кода
    \begin{itemize}
      \item пакет - bg.drow.spellbook
      \item директория bg\\drow\\spellbook
    \end{itemize}
  \item Не съществува йерархична организация на пакетите
  \end{itemize}
\end{frame}

\begin{frame}{Пакетиране на Java код}
  \transdissolve
  \begin{itemize}
  \item jar архив - bytecode + мета информация, компресирани със zip
  \item Добавяне на jar архиви в клас пътя на виртуалната машина
  \item Изпълнение на код пакетиран в jar архив
    java -jar somejar.jar MainClass
  \end{itemize}
\end{frame}

\begin{frame}{Класпът в Java(classpath)}
  \transdissolve
  \begin{itemize}
  \item Указва местата, където виртуалната машина търси класове, при
    изпълнението на дадена програма
  \item може да се задава като параметър от командния ред или като
    настройка на средата
  \item Източник на много проблеми
  \end{itemize}
\end{frame}


\begin{frame}{Пример}
  \transdissolve
  В UNIX(Linux, Solaris, BSD)
  java -cp .:dir2:dir3 MyClass
  export CLASSPATH=.:dir2:dir3


  В Windows:
  Java -cp .;dir2;dir3; MyClass
  set CLASSPATH=.;dir2;dir3
\end{frame}

\begin{frame}{Документиране на класове с javadoc}
  \transdissolve
  \begin{itemize}
  \item Разглеждане на javadoc документация
  \item Документира се само публичния интерфейс на едно API
  \item Стандартни javadoc анотации
  \item Работа с инструмента javadoc
  \end{itemize}
\end{frame}

\begin{frame}{Работа с javadoc}
  \transdissolve
  \begin{itemize}
  \item   javadoc -d docdir package – създава html
    документация за пакета package в
    директорията docdir
  \item  javadoc -d docdir package1, package2 –
    създава документация за няколко пакета

  \end{itemize}
\end{frame}

\begin{frame}{Седем прости правила за създаване на добри класове}
  \transdissolve
  \begin{enumerate}
    \item Правете всички полета private
    \item Инициализирайте изрично всички полета
    \item Не използвайте много примитивни полета
    \item Не слагайте get/set методи, ако нямате нужда от тях
    \item Дефинирайте класовете по стандартен начин
    \item Раздробявайте класовете с много отговорности на по-малки
    \item Избирайте смислени име за класовете, полетата и методите
  \end{enumerate}
\end{frame}


\begin{frame}{Упражнение}
  \transdissolve
  
\end{frame}


\section*{Заключение}

\begin{frame}{Заключение}
  \transdissolve
  % Keep the summary *very short*.
  \begin{itemize}
  \item
    Класовете са шаблоните, по които изграждаме обекти.
  \item
    Обектите са градивните блокове на типичната Java програма.
  \item
    Проектирането на класове е критичен процес, при който трябва да
    бъдат съблюдавани утвърдените практики.
  \end{itemize}
  
  % The following outlook is optional.
  \vskip0pt plus.5fill
  \begin{itemize}
  \item
    Следващият път:
    \begin{itemize}
    \item
      Наследяване
    \item
      Полиморфизъм
    \end{itemize}
  \end{itemize}
\end{frame}

\begin{frame}{Въпроси}
  \transdissolve
  \begin{center}
    \LARGEТук е момента да зададете вашите въпроси! :-)
  \end{center}
\end{frame}

\begin{frame}{Край}
  \transdissolve
  \begin{center}
    \LARGEБлагодаря Ви за вниманието!
  \end{center}
\end{frame}

\end{document}

%%% Local Variables: 
%%% mode: latex
%%% TeX-master: t
%%% End: 
