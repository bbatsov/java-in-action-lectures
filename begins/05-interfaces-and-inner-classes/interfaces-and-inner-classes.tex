\documentclass{beamer}

% Copyright 2010 Drow Ltd.
% 
% In principle, this file can be redistributed and/or modified under
% the terms of the GNU Public License, version 2.
% 
% However, this file is supposed to be a template to be modified
% for your own needs. For this reason, if you use this file as a
% template and not specifically distribute it as part of a another
% package/program, I grant the extra permission to freely copy and
% modify this file as you see fit and even to delete this copyright
% notice. 
\mode<presentation>
{
  \usetheme[titleline=true,
  alternativetitlepage=true,
  titlepagelogo=images/Java_logo]{Torino}
  \usecolortheme{nouvelle}
  \beamertemplatenavigationsymbolsempty
}

\usepackage{times}
\usepackage[utf8]{inputenc}
\usepackage[english,bulgarian]{babel}
\usepackage[T2A]{fontenc}

\usepackage{listings}
\lstset{language=Java,
  captionpos=b,
  tabsize=4,
  keywordstyle=\color{blue},
  commentstyle=\color{gray},
  stringstyle=\color{green},
  numbers=left,
  breaklines=true,
  showstringspaces=false,
  basicstyle=\ttfamily,
  emph={label},
  frame=shadowbox, 
  rulesepcolor=\color{blue},
  columns=fixed}

\title{Интерфейси и вътрешни класове}

\author{инж. Божидар ~Бацов}

\institute{Drow Ltd.}

\date{23.11.2010}

\subject{Talks}
% This is only inserted into the PDF information catalog. Can be left
% out. 

\begin{document}

\begin{frame}
  \titlepage
\end{frame}

\begin{frame}{Съдържание}
  \tableofcontents[pausesections]
\end{frame}

\section{Интерфейси}

\begin{frame}{Интерфейс}
  \begin{itemize}
  \item
    Наборът от методи, които един клас предлага
  \item
    Езикова конструкция, която съдържа контракт(набор
    от задължения)
  \item Класовете могат да изпълнят контракта(да имплементират
    интерфейса)
  \item Обикновено съдържа само методи
  \item Методите са с ниво на достъп \textbf{public} и са \textbf{абстрактни}
  \end{itemize}
\end{frame}

\begin{frame}{Диаграма}
  \transdissolve
  
\end{frame}

\begin{frame}[fragile]
  \frametitle{Дефиниция на интерфейс}
  \transdissolve
\begin{lstlisting}
interface SomeInterface {
  Type field1; // bad style
  Type field2;
  ...
  Type method1();
  Type method2();
}
\end{lstlisting}
\end{frame}

\begin{frame}[fragile]
  \frametitle{Имплементиране на интерфейс}
  \transdissolve
\begin{lstlisting}
SomeClass implements SomeInterface {
  ...  
  @Override
  public Type method1() {...};
  @Override
  public Type method2() {...}
}
\end{lstlisting}
\end{frame}

\begin{frame}{Особености на интерфейсите}
  \transdissolve
  \begin{itemize}
  \item Всички методи в тях са абстрактни
  \item Един клас може да имплементира повече от един интерфейс
  \item Интерфейсът \textbf{НЕ Е} клас - не може създавате обекти от
    интерфейс
  \item Могат да бъдат декларирани променливи от интерфейсен тип
  \item Работят с instanceof оператора
  \item Не е желателно да имат полета
  \item Абстрактен клас може да имплементира интерфейс напълно или
    частично
  \end{itemize}
\end{frame}

\begin{frame}{Особености на интефейсите}
  \transdissolve
  \begin{itemize}
  \item Не могат да съдържат instance полета
  \item Не могат да съдържа статични методи
  \item Всички полета в един интерфейс са на практика константи(public
    static final)
  \item Един интерфейс може да разшири(наследи) друг
  \item Интерфейси без методи се използват като маркери(tags)
  \end{itemize}
\end{frame}

\begin{frame}[fragile]
\frametitle{Сравняване на обекти с интерфейса Comparable}
\transdissolve
\begin{lstlisting}
public interface Comparable<T> {
  int compareTo(T other);
}
\end{lstlisting}
\begin{itemize}
  \item Използва се за сравняване на обекти с естествена подредба
  \item Генеричен интерфейс от Java 5
\end{itemize}

\end{frame}

\begin{frame}[fragile]
  \frametitle{Comparable - пример}
  \transdissolve
\begin{lstlisting}
  
\end{lstlisting}
\end{frame}

\begin{frame}{Клониране обекти}
  \transdissolve
  \begin{itemize}
  \item метода clone() на класа Object
  \item интерфейсът Cloneable() - интерфейс маркер
  \item плитко клониране
  \item дълбоко клониране
  \item имплементиране на clone()
  \item клониране на масиви
  \end{itemize}
\end{frame}


\begin{frame}{Клониране - диаграма}
  \transdissolve
  
\end{frame}

\begin{frame}{Обработка на събития}
  \transdissolve
  \begin{itemize}
  \item Събитие - изтекъл таймер, натиснат бутон
  \item Callback
  \item Реализация на callback механизъм с интерфейси
  \end{itemize}
\end{frame}

\begin{frame}[fragile]
  \frametitle{Пример}
  \transdissolve
\begin{lstlisting}
  
\end{lstlisting}
\end{frame}

\section{Вътрешни класове}

\begin{frame}{Вътрешни класове}
  \transdissolve
  \begin{itemize}
  \item Влагане да дефиницията на един клас в друг
  \item Реализирани са на ниво компилатор
  \item Позволяват достъп до членовете на класа, в който са вложени
  \item Видове
    \begin{itemize}
      \item стандартни
      \item локални 
      \item анонимни
      \item статични
    \end{itemize}

  \end{itemize}
\end{frame}

\begin{frame}{Стандартни вътрешни класове}
  \transdissolve
  \begin{itemize}
  \item Дефинирани са в друг клас на нивото на полетата и методите му
  \item Имат достъп до полетата и методите на външния клас
  \item Имат скрита референция към външния клас
  \item Всяка инстанция от външния клас носи дефиницията на вътрешния
  \end{itemize}
\end{frame}

\begin{frame}[fragile]
  \frametitle{Пример}
  \transdissolve
\begin{lstlisting}
  
\end{lstlisting}
\end{frame}

\begin{frame}{Локалния вътрешни класове}
  \transdissolve
  \begin{itemize}
  \item Дефинирани са в тялото на метод на външния клас
  \item Имат достъп до локалните променливи в метода
  \item Не са видими извън метода, в който са дефинирани
  \end{itemize}
\end{frame}

\begin{frame}[fragile]
  \frametitle{Пример}
  \transdissolve
\begin{lstlisting}
  
\end{lstlisting}
\end{frame}

\begin{frame}{Анонимни вътрешни класове}
  \transdissolve
  \begin{itemize}
  \item Разновидност на локалните класове
  \item Създава се обект от клас дефиниран след оператора за
    присвояване
  \item Обикновено анонимните класове имплементират някой интерфейс
  \end{itemize}
\end{frame}

\begin{frame}[fragile]
  \frametitle{Пример}
  \transdissolve
\begin{lstlisting}
  
\end{lstlisting}
\end{frame}

\begin{frame}{Статични вътрешни класове}
  \transdissolve
  \begin{itemize}
  \item Еквивалентни на стандартните, но без референция към външния
    клас
  \item Служат като допълнително пространство на имената
  \end{itemize}
\end{frame}

\begin{frame}[fragile]
  \frametitle{Пример}
  \transdissolve
\begin{lstlisting}
  
\end{lstlisting}
\end{frame}


\section*{Заключение}

\begin{frame}{Заключение}
  \transdissolve
  % Keep the summary *very short*.
  \begin{itemize}
  \item
    Интерфейсите са проста алтернатива на множественото наследяване.
  \item
    Вътрешните класове ви дават възможност да изразите по-прецизно
    връзките между класовете ви.
  \end{itemize}
  
  % The following outlook is optional.
  \vskip0pt plus.5fill
  \begin{itemize}
  \item
    Следващият път:
    \begin{itemize}
    \item
      Разработка на графични потребителски интерфейси със Swing
    \item
      Повече примери, по-малко общи приказки
    \end{itemize}
  \end{itemize}
\end{frame}

\begin{frame}{Въпроси}
  \transdissolve
  \begin{center}
    \LARGEТук е момента да зададете вашите въпроси! :-)
  \end{center}
\end{frame}

\begin{frame}{Край}
  \transdissolve
  \begin{center}
    \LARGEБлагодаря Ви за вниманието!
  \end{center}
\end{frame}

\end{document}

%%% Local Variables: 
%%% mode: latex
%%% TeX-master: t
%%% End: 
