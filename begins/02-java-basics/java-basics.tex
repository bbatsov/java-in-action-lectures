\documentclass{beamer}

% Copyright 2010 Drow Ltd.
% 
% In principle, this file can be redistributed and/or modified under
% the terms of the GNU Public License, version 2.
% 
% However, this file is supposed to be a template to be modified
% for your own needs. For this reason, if you use this file as a
% template and not specifically distribute it as part of a another
% package/program, I grant the extra permission to freely copy and
% modify this file as you see fit and even to delete this copyright
% notice. 

\mode<presentation>
{
  \usetheme[titleline=true,
            alternativetitlepage=true,
            titlepagelogo=images/Java_logo]{Torino}
  \usecolortheme{nouvelle}
  \beamertemplatenavigationsymbolsempty
}

\usepackage{helvet}
\usepackage[utf8]{inputenc}
\usepackage[english,bulgarian]{babel}
\usepackage[T2A]{fontenc}

\usepackage{listings}
\lstset{language=Java,
        captionpos=b,
        tabsize=4,
        keywordstyle=\color{blue},
        commentstyle=\color{gray},
        stringstyle=\color{green},
        numbers=left,
        breaklines=true,
        showstringspaces=false,
        basicstyle=\ttfamily,
        emph={label},
        frame=shadowbox, 
        rulesepcolor=\color{blue},
        columns=fixed}

\title{Основи на Java}

\subtitle
{Структура на Java програма. Поток на изпълнение. Управляващи
  конструкции. Примитивни типове данни и масиви.}

\author[инж. Божидар Бацов]{инж. Божидар ~Бацов}

\institute{Drow Ltd.}

\date{26.10.2010}

\subject{Talks}
% This is only inserted into the PDF information catalog. Can be left
% out. 

% Delete this, if you do not want the table of contents to pop up at
% the beginning of each subsection:
\AtBeginSubsection[]
{
  \begin{frame}<beamer>{Съдържание}
    \tableofcontents[currentsection,currentsubsection]
  \end{frame}
}


% If you wish to uncover everything in a step-wise fashion, uncomment
% the following command: 

% \beamerdefaultoverlayspecification{<+->}

\begin{document}

% \lstset{language=Java}

\begin{frame}
  \titlepage
\end{frame}

\begin{frame}{Съдържание}
  \tableofcontents
  % You might wish to add the option [pausesections]
\end{frame}


% Since this a solution template for a generic talk, very little can
% be said about how it should be structured. However, the talk length
% of between 15min and 45min and the theme suggest that you stick to
% the following rules:  

% - Exactly two or three sections (other than the summary).
% - At *most* three subsections per section.
% - Talk about 30s to 2min per frame. So there should be between about
% 15 and 30 frames, all told.

\section{Структура и поток на изпълнение на Java програма}

\subsection{Структура на Java програма}

\begin{frame}{Градивни блокове}
  \begin{itemize}
  \item твърдения(statements)
  \item методи
  \item полета
  \item класове
  \item пакети
  \item коментари
  \end{itemize}
\end{frame}


\begin{frame}{Основни положения}  
  \begin{itemize}
    \item Целият код на програмата живее в класове
    \item Входната точка на всяка програма е клас с main метод
    \item Програмите са чувствителни към малки и големи букви
    \item Твърденията се терминират с
    \item Коментарите се игнорират от компилатора
  \end{itemize}
\end{frame}


\begin{frame}[fragile]
 \frametitle{Пример}
 \begin{lstlisting}
public class SimpleClass {
  public static void main(String[] args) {
    System.out.println("Hello!");
  }
}
  \end{lstlisting}
\end{frame}

\subsection{Поток на изпълнение}
\begin{frame}{Поток на изпълнение}
  
  \begin{itemize}
    \item Твърденията в една Java програма се изпълняват последователно
    \item Коментари
      
      \begin{itemize}
        \item C стил /* ... */ - за коментар с произволен размер
        \item C++ стил // - коментар до края на реда
        \item javadoc /** ... */ - специален коментар
      \end{itemize}

  \end{itemize}

\end{frame}

\begin{frame}[fragile]
  \frametitle{Пример}
  \begin{lstlisting}
    public class ExecutionFlowExample1 {
      public static void main(String[] args) {
        System.out.println("Kiril"); /* will be printed first */
        System.out.println("Bozhidar");
      }
    }
  \end{lstlisting}
\end{frame}

\begin{frame}[fragile]
  \frametitle{Пример}
  \begin{lstlisting}
    public class ExecutionFlowExample2 {
      public static void main(String[] args) {
        System.out.println("Bozhidar");  /* will be printed first */
        System.out.println("Kiril");
      }
    }
  \end{lstlisting}
\end{frame}


\section{Примитивни типове данни}


\begin{frame}{Целочислени типове}
  \begin{itemize}
  \item int – 4 байта : -2,147,483,648 до
    2,147,483, 647
  \item short – 2 байта : -32,768 до 32,767
  \item long – 8 байта – суфикс L
  \item byte – 1 байт : -128 до 127
  \end{itemize}

  Могат да се използват 8, 10 и 16тини
  бройни системи
\end{frame}


\begin{frame}{Типове с плаваща запетая}
  \begin{itemize}
  \item   float – 4 байта – 6-7 значими знака след
    десетичната запетая – суфикс F
    \item double – 8 байта – 15 значими знака
    след десетичната запетая – суфикс D
  \end{itemize}
\end{frame}


\begin{frame}{Символен тип}
  \begin{itemize}
  \item char
  \end{itemize}
\end{frame}


\begin{frame}{Булев тип}
  \begin{itemize}
  \item boolean - има две дискретни стойности true и false
  \item 0 и null не се считат за false
  \item всички логически изрази имат резултат от тип boolean
  \end{itemize}
\end{frame}


\begin{frame}{Променливи}
  \begin{itemize}
  \item Всяка променлива трябва да има
    зададен тип и име

  \item Името трябва да започва с буква или \_
    и да съдържа само букви, цифри и \_

  \item Имената на променливите трябва да спазват конвенцията camelCase
  
  \item Променливите трябва да бъдат инициализирани преди да бъдат използвани!
  \end{itemize}
\end{frame}

\begin{frame}[fragile]
  \frametitle{Пример}
\begin{lstlisting}
  int i; // declaration
  i = 5; // initialization
  int j = 5; // declaration + initialization
  int k, l = 13;
  long daysLeftUntilEndOfTheYear;
  char mostCommonlyUsedCharacter;
  int students;
  System.out.println(students); // ERROR - variable not initialized
\end{lstlisting}
\end{frame}

\begin{frame}{Свойства на променливите}
  \begin{itemize}
  \item Могат да бъдат декларирани навсякъде(преди да бъдат
    използвани)
  \item Променливите имат обхват(scope)
  \item На променливите(които не са маркирани като final) може да бъде
    присвоена нова стойност
  \end{itemize}
\end{frame}

\begin{frame}{Оператори}
  \begin{itemize}
  \item + - събиране
  \item - - изваждане
  \item * - умножение
  \item / - делене – целочислено, ако и двете
    числа са цели числа, иначе с плаваща
    запетая
  \item \% - остатък при делене
  \end{itemize}
\end{frame}

\begin{frame}[fragile]
  \frametitle{Пример}
\begin{lstlisting}
  int i = 5 + 8; // initialize i to 13
  i = i + 5; // add 5 to i, i becomes 18
  i += 5; // add 5 to i, i becomes 23
  int p = i + 8; // initialize to i plus 8, i.e. 31
  int q = i + p; // initialize to i plus p, i.e. 54
  int t = q * 7; // initialize to q * 7, i.e. 378
\end{lstlisting}
\end{frame}


\begin{frame}{Оператори за увеличаване и намаляване с 1}
  \begin{itemize}
  \item   Могат да бъдат прилагани само върху
    променливи от целочислен тип.
    Променят стойността на променливата.

   \item ++var – увеличава стойността на var с 1
    преди да бъде ползвана

   \item --var – намалява стойността на var с 1,
    преди да бъде ползвана

  \end{itemize}
\end{frame}

\begin{frame}[fragile]
  \frametitle{Пример}
\begin{lstlisting}
  int i = 5;
  ++i;
  System.out.println(i); // prints 6
  System.out.println(++i); // prints 7
  System.out.println(i); // prints 7
  System.out.println(--i); // prints 6
  System.out.println(i); // prints 6
\end{lstlisting}
\end{frame}


\begin{frame}{Оператори за увеличаване и намаляване с 1}
  \begin{itemize}
  \item var++ - увеличава с 1 стойността на
    променливата, след като стойността е
    използвана
  \item var-- - намалява с 1 стойността на
    променливата, след като стойността е
    използвана
  \end{itemize}
\end{frame}

\begin{frame}[fragile]
  \frametitle{Пример}
\begin{lstlisting}
  int i = 5;
  i++;
  System.out.println(i); // prints 6
  System.out.println(i++); // prints 6
  System.out.println(i); // prints 7
  System.out.println(i--); // prints 7
  System.out.println(i); // prints 6
\end{lstlisting}
\end{frame}


\begin{frame}{Оператори за сравнение}
  Ползват се върху сравними примитивни
  типове и връщат булева стойност като резултат.
  
  \begin{itemize}
  \item == - проверка за равенство
  \item != - проверка за различие
  \item $<$  - по-малко
  \item $<$= - по-малко или равно
  \item $>$  - по-голямо
  \item $>$= - по-голямо или равно
  \end{itemize}

\end{frame}


\begin{frame}{Логически оператори}
  \begin{itemize}
  \item   Логически израз се изчислява, докато
    не е сигурна неговата стойност (short
    circuit)
  \item \&\& - логическо И
  \item || - логическо ИЛИ
  \item ! - логическо отрицание

  \end{itemize}
\end{frame}


\begin{frame}{Таблица на истинност}
  
\end{frame}

\begin{frame}[fragile]
  \frametitle{Пример}
\begin{lstlisting}
  int x = 2;
  x != 0 && 10 / x > 5; // false
  !(x > 5) && x != 3; // true
  x > 5 || x == 8; // false
  int y = 0;
  (x <5 || x > 9) && y != 0 && x + y != 0; // false
\end{lstlisting}
\end{frame}


\begin{frame}{Автоматично преобразуване между примитивни числени типове}
  
\end{frame}


\begin{frame}[fragile]
  \frametitle{Изрично преобразуване}
  \begin{itemize}
  \item   Когато искаме преобразования водещи
    до загуба на информация
  \item ( тип-към-който-преобразуваме ) променлива
  \end{itemize}
  \begin{lstlisting}
   double x = 6.79;
   int i = (int) x; // i == 6
   long t = 8; int q = (int) t;
  \end{lstlisting}
\end{frame}

\begin{frame}{Блокове}
  \begin{itemize}
  \item Позволява да бъдат групирани няколко
    твърдения
   \item Дефиницията на блок започва с \{ и
      завършва с \} - между тях има
    твърдения
   \item Имат собствен обхват
   \item Могат да бъдат влагани един в друг
  \end{itemize}
\end{frame}

\begin{frame}[fragile]
  \frametitle{Блокове - пример}
\begin{lstlisting}
  public static void main(String[] args) {
   int n;
   ...
   {
     int k;
    . . . } // k is only defined up to here
}
\end{lstlisting}
\end{frame}

\begin{frame}[fragile]
  \frametitle{Блокове - пример}
\begin{lstlisting}
  public static void main(String[] args) {
   int n;
   {
      int k;
      int n; // error--can't redefine n in inner
             // block
    . . .} }

\end{lstlisting}
\end{frame}

\section{Управляващи конструкции}

\subsection{Условни конструкции}

\begin{frame}{Условна конструкция if}
  \begin{itemize}
  \item Позволява условното изпълнение на код при изпълнено условие
  \item   if (condition) statement
  \item  condition – логически израз
    –
  \item  statement – твърдение (или множество
    –
    твърдения в блок), което да бъде
    изпълнено, ако условието е вярно

  \end{itemize}
\end{frame}


\begin{frame}[fragile]
  \frametitle{if - пример}
\begin{lstlisting}
  if (performance > 5){
    salary+=500;
    bonus = 100;
}

\end{lstlisting}
\end{frame}

\begin{frame}{Условна конструкция if-else}
  \begin{itemize}
  \item   if (condition) statement1 else statement2
    \item condition – логически израз
    –
    \item statement1 – твърдение, което да бъде
    –
    изпълнено, ако условието е вярно
    \item statement2 – твърдение, което да бъде
    –
    изпълнено, ако условието е невярно

  \end{itemize}
\end{frame}

\begin{frame}[fragile]
  \frametitle{if-else - пример}
\begin{lstlisting}
  if (performance > 5){
    salary += 500;
    bonus = 100;
} else {
    bonus = 0;
}
\end{lstlisting}
\end{frame}

\begin{frame}[fragile]
  \frametitle{if-else-if-else}
\begin{lstlisting}
  If (performance > 5) {
 salary += 500;
 bonus = 100;
} else if (performance > 2) {
   salary += 100;
   bonus = 0;
} else { salary = 0; bonus = 0; } // you're fired

\end{lstlisting}
\end{frame}

\begin{frame}{Множествен избор със switch}
  \begin{itemize}
  \item    Множествен избор - switch
    Switch – множествен избор – може да се
    използва с byte, short, char, int

    switch (variable) {
      case VALUE : statements;
      case VALUE : statements;
      default : statements;
    }

  \end{itemize}
\end{frame}

\begin{frame}[fragile]
  \frametitle{switch - пример}
\begin{lstlisting}
int month = 8;
switch (month) {
case 1: System.out.println("January");  break;
...
case 12:
System.out.println("December"); break;
default: System.out.println("Invalid month."); break;
}

\end{lstlisting}
\end{frame}

\begin{frame}[fragile]
  \frametitle{switch - пример}
\begin{lstlisting}
int startFrom = 6;
switch (month) {
case 1: System.out.println(1);
...
case 6: System.out.println(6);
case 10: System.out.println(10); break;
default: System.out.println("Invalid number."); break;
}
\end{lstlisting}
\end{frame}

\subsection{Цикли}

\begin{frame}{Цикъл while}
  \begin{itemize}
  \item Синтаксис - while (condition) \{statement(s);\}
    \begin{itemize}
      \item условие(condition) определя дали ще изпълни тялото на цикъла
      \item тялото на цикъла се състои от едно или повече твърдения
    \end{itemize}

  \item Семантика - цикълът while може и ДА НЕ СЕ изпълни нито веднъж
  \item Поток на изпълнение - след изпълнение на тялото контрола се
    прехвърля в условието на цикъла
  \end{itemize}
\end{frame}

\begin{frame}[fragile]
  \frametitle{Цикъл while - пример}
\begin{lstlisting}
int money = 20;
int beersDrank = 0;
while (money > 5) {
    money -= 3; // buy beer
    beersDrank++;
}
System.out.print("We've drunk " +  beersDrank + " beers.");
\end{lstlisting}
\end{frame}

\begin{frame}{Цикъл do-while}
  \begin{itemize}
  \item Синтаксис - do \{statements\} while (condition)
  \item Семантика - цикълът do-while се изпълнява винаги поне веднъж
  \end{itemize}
\end{frame}

\begin{frame}[fragile]
  \frametitle{Цикъл do-while - пример}
\begin{lstlisting}
int count = 1;
do {
    System.out.println("Count is: " + count);
    count++;
} while (count <= 11);
\end{lstlisting}
\end{frame}

\begin{frame}{Цикъл for}
  \begin{itemize}
  \item   for ( initialization; condition; update )  {statement}
    \item initialization инициализираме променливи
    \item condition логически израз
    \item update – изпълнява се, докато
    условието е вярно, преди statement

    \item statement – изпълнява се, докато
    условието е вярно, след update

  \end{itemize}
\end{frame}

\begin{frame}[fragile]
  \frametitle{Цикъл for - пример}
\begin{lstlisting}
for (int i = 1; i <= 10; i++) {
  System.out.println(i);
}
// i no longer defined
int i;
for (i = 1; i <= 10; i++) {
  System.out.println(i);
}
// i still defined
\end{lstlisting}
\end{frame}

\begin{frame}[fragile]
  \frametitle{Прекратяване на цикъл преждевременно}
  \transdissolve
  \begin{block}{break}
    break прекратява текущия цикъл. Може
    да се ползва в for, while, do-while и
    switch.
  \end{block}
\begin{lstlisting}
int i;
for( i = startNumber; I < startNumber + 5; i++){
  if (I % 5 == 0) break; 
}
System.out.println("closest number : " + i);
\end{lstlisting}
\end{frame}

\begin{frame}[fragile]
  \frametitle{Преминаване към следващата итерация}
  \transglitter[direction=90]
  \begin{block}{continue}
    преминаване към следваща
    итерация в цикъл. Може да се
    използва в for, while, do-while цикли
  \end{block}
\begin{lstlisting}
int numbersNotDivideableByThree = 0;
for(int i = startValue; i<1000; i++){
     if(i%3==0) continue;
     numbersNotDivideableByThree++; }
\end{lstlisting}
\end{frame}

\section{Масиви}


\begin{frame}{Масиви}
  \begin{block}{Масив}
    съдържа елементи от даден
    тип. Може да бъде многоизмерен
    (матрица). Достъп по индекс,
    индексацията започва от 0 (НУЛА).
    Чрез .length взимаме дължината на
    масива.
  \end{block}
\end{frame}

\begin{frame}[fragile]
  \frametitle{Масив - пример}
\begin{lstlisting}
  String[] names = { "Pesho", "Gosho", "Ivan" };
for(int i=0; i<names.length; i++){
  System.out.println("Name : " + names[i]);
}
\end{lstlisting}
\end{frame}

\begin{frame}[fragile]
  \frametitle{Масив - пример}
\begin{lstlisting}
  int[] numbers = { 1, 8, 10, 33 };
long sum = 0;
for(int i = 0; i < numbers.length; i++) {
    sum += numbers[i];
}
System.out.println("Sum : " + sum);
\end{lstlisting}
\end{frame}

\begin{frame}{Цикъл for-each}
  \begin{itemize}
  \item             For-each цикъл
    for (array-type variable : array) statement

    \item Array-type – тип на променливата
    –variable, трябва да съвпада с типа на
    елементите в масива
    \item Variable – променлива, съдържаща
    текущия елемент от масива
    \item Array – масив, от който да бъдат
    взимани елементи

  \end{itemize}
\end{frame}

\begin{frame}[fragile]
  \frametitle{Цикъл for-each - пример}
\begin{lstlisting}
  
\end{lstlisting}
\end{frame}


\begin{frame}{Упражнение}
  
\end{frame}


\section*{Заключение}

\begin{frame}{Заключение}

  % Keep the summary *very short*.
  \begin{itemize}
  \item
    Java \alert{е много повече от език за програмиране}.
  \item
    JVM \alert{е целевата среда за изпълнение} на Java приложенията, а
    не физическата процесорна микроархитектура.
  \item
    За пълноценна работа с езикът и платформата Java човек трябва да
    се запознае с доста инструменти.
  \end{itemize}
  
  % The following outlook is optional.
  \vskip0pt plus.5fill
  \begin{itemize}
  \item
    Следващият път:
    \begin{itemize}
    \item
      Основния положения в езикът Java
    \item
      Повече примери, по-малко общи приказки
    \end{itemize}
  \end{itemize}
\end{frame}


\begin{frame}{Въпроси}
  \begin{center}\LARGEТук е момента да зададете вашите въпроси! :-)\end{center}
\end{frame}


\begin{frame}{Край}

  \begin{center}
    \LARGEБлагодаря Ви за вниманието!
  \end{center}
  
\end{frame}

\end{document}

%%% Local Variables: 
%%% mode: latex
%%% TeX-master: t
%%% End: 
