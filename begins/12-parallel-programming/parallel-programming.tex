\documentclass{beamer}

% Copyright 2010 Drow Ltd.
% 
% In principle, this file can be redistributed and/or modified under
% the terms of the GNU Public License, version 2.
% 
% However, this file is supposed to be a template to be modified
% for your own needs. For this reason, if you use this file as a
% template and not specifically distribute it as part of a another
% package/program, I grant the extra permission to freely copy and
% modify this file as you see fit and even to delete this copyright
% notice. 
\mode<presentation>
{
  \usetheme[titleline=true,
  alternativetitlepage=true,
  titlepagelogo=images/Java_logo]{Torino}
  \usecolortheme{nouvelle}
  \beamertemplatenavigationsymbolsempty
}

\usepackage{times}
\usepackage[utf8]{inputenc}
\usepackage[english,bulgarian]{babel}
\usepackage[T2A]{fontenc}

\usepackage{listings}
\lstset{language=Java,
  captionpos=b,
  tabsize=4,
  keywordstyle=\color{blue},
  commentstyle=\color{gray},
  stringstyle=\color{green},
  numbers=left,
  breaklines=true,
  showstringspaces=false,
  basicstyle=\ttfamily,
  emph={label},
  frame=shadowbox, 
  rulesepcolor=\color{blue},
  columns=fixed}

\title{Паралелно програмиране}

\author{инж. Божидар ~Бацов}

\institute{Drow Ltd.}

\date{26.10.2010}

\subject{Talks}
% This is only inserted into the PDF information catalog. Can be left
% out. 

% If you wish to uncover everything in a step-wise fashion, uncomment
% the following command: 

% \beamerdefaultoverlayspecification{<+->}


\begin{document}

\begin{frame}
  \titlepage
\end{frame}

\begin{frame}{Съдържание}
  \tableofcontents
  % You might wish to add the option [pausesections]
\end{frame}


% Since this a solution template for a generic talk, very little can
% be said about how it should be structured. However, the talk length
% of between 15min and 45min and the theme suggest that you stick to
% the following rules:  

% - Exactly two or three sections (other than the summary).
% - At *most* three subsections per section.
% - Talk about 30s to 2min per frame. So there should be between about
% 15 and 30 frames, all told.

\section{Хардуер и софтуер}

\subsection{Хардуер}

\begin{frame}{Хардуерната еволюция}
  \begin{itemize}
  \item Войната на GHz приключи
  \item Повече ядра срещу по-високи тактови честоти
  \item Паралелни компютри
    \begin{itemize}
      \item Мултикомпютър
      \item Мултипроцесор
    \end{itemize}
  \end{itemize}
\end{frame}

\begin{frame}{Един процесор, много приложения}
  \transdissolve
  \begin{itemize}
  \item Един процесор в даден момент може да изпълнява само едно
    приложение(много грубо казано)
  \item Но на еднопроцесорен компютър работят едно временно
    \begin{itemize}
      \item браузър
      \item текстов редактор
      \item пощенски клиент
      \item музикален плеър 
      \item офис пакет
      \item терминален емулатор
      \item офис пакет
    \end{itemize}

  \end{itemize}
\end{frame}

\begin{frame}{Малко теория на операционните системи}
  \transdissolve
  \begin{itemize}
  \item Диспечър(scheduler)
    \begin{itemize}
      \item разпределя ресурсите на операционна система между
        програмите, която тя изпълнява
      \item по определен алгоритъм предоставя на всяка програма кванти
        процесорно време, през които те работят
      \item един процес се изпълнява, всички други спят
      \item създава илюзия за паралелизъм
    \end{itemize}

  \end{itemize}
\end{frame}

\begin{frame}{Аспекти на едно приложение}
  \transdissolve
  \begin{itemize}
  \item Приоритет
    \begin{itemize}
      \item процесите с по-голям процес получават от диспечъра повече
        процесорно време
    \end{itemize}

  \item Комбинация от процесорно време и входно изходни операции

  \item За истински паралелизъм са необходими повече от един
    процесор(или процесор с повече от едно ядро или технология като HT
    на Intel)
  \end{itemize}
\end{frame}

\begin{frame}{Единици за програмно изпълнение}
  \transdissolve
  \begin{block}{Процеси}
     – няколко независими процеса,
     които се синхронизират посредством
    съобщения, които обменят по определен
    комуникационен канал
  \end{block}
  \begin{block}{Нишки}
    – потоци на изпълнение, които се
    изпълняват в общо адресно
    пространство и се синхронизират
    посредством променливите, които
    споделят
  \end{block}
\end{frame}

\begin{frame}{Нишки в Java}
  \transdissolve
  \begin{itemize}
  \item класът Thread
    \begin{itemize}
      \item обектна капсулация на нишка
      \item притежава полезни методи като start()
    \end{itemize}
  \item интерфейсът Runnable
    \begin{itemize}
      \item шаблон за създаване на изпълними в нишка обекти
    \end{itemize}
  \item приложение на нишки
    \begin{itemize}
      \item времеемки операции
      \item операции, които се изпълняват във фона на приложението
    \end{itemize}
  \end{itemize}
\end{frame}

\begin{frame}{Синхронизация на нишки}
  \transdissolve
  \begin{itemize}
  \item Синхронизационни обекти
    \begin{itemize}
      \item семафори
      \item locks
      \item mutexes
    \end{itemize}
    \item Ключовата дума на synchronized
      \begin{itemize}
        \item синхронизирани методи
        \item синхронизирани блокове
      \end{itemize}

    \item deadlocks
  \end{itemize}
\end{frame}

\begin{frame}{Executors framework}
  \transdissolve
  \begin{itemize}
  \item Въведена в Java 5
  \item Живее в пакета java.util.concurrent
  \item Предлага по-високо ниво на абстракция от нишките - задачи
  \item Thread pools
  \item Scheduled execution
  \end{itemize}
\end{frame}

\begin{frame}{Нишки в Swing}
  \transdissolve
  \begin{itemize}
  \item Event dispatch thread
  \item SwingWorker
  \end{itemize}
\end{frame}


\section*{Заключение}

\begin{frame}{Заключение}
  \transdissolve
  % Keep the summary *very short*.
  \begin{itemize}
  \item
    Java \alert{е много повече от език за програмиране}.
  \item
    JVM \alert{е целевата среда за изпълнение} на Java приложенията, а
    не физическата процесорна микроархитектура.
  \item
    За пълноценна работа с езикът и платформата Java човек трябва да
    се запознае с доста инструменти.
  \end{itemize}
  
  % The following outlook is optional.
  \vskip0pt plus.5fill
  \begin{itemize}
  \item
    Следващият път:
    \begin{itemize}
    \item
      Основния положения в езикът Java
    \item
      Повече примери, по-малко общи приказки
    \end{itemize}
  \end{itemize}
\end{frame}

\begin{frame}{Въпроси}
  \transdissolve
  \begin{center}
    \LARGEТук е момента да зададете вашите въпроси! :-)
  \end{center}
\end{frame}

\begin{frame}{Край}
  \transdissolve
  \begin{center}
    \LARGEБлагодаря Ви за вниманието!
  \end{center}
\end{frame}

\end{document}

%%% Local Variables: 
%%% mode: latex
%%% TeX-master: t
%%% End: 
