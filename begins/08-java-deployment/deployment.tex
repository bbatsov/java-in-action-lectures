\documentclass{beamer}

% Copyright 2010 Drow Ltd.
% 
% In principle, this file can be redistributed and/or modified under
% the terms of the GNU Public License, version 2.
% 
% However, this file is supposed to be a template to be modified
% for your own needs. For this reason, if you use this file as a
% template and not specifically distribute it as part of a another
% package/program, I grant the extra permission to freely copy and
% modify this file as you see fit and even to delete this copyright
% notice. 
\mode<presentation>
{
  \usetheme[titleline=true,
  alternativetitlepage=true,
  titlepagelogo=images/Java_logo]{Torino}
  \usecolortheme{nouvelle}
  \beamertemplatenavigationsymbolsempty
}

\usepackage{times}
\usepackage[utf8]{inputenc}
\usepackage[english,bulgarian]{babel}
\usepackage[T2A]{fontenc}

\usepackage{listings}
\lstset{language=Java,
  captionpos=b,
  tabsize=4,
  keywordstyle=\color{blue},
  commentstyle=\color{gray},
  stringstyle=\color{green},
  numbers=left,
  breaklines=true,
  showstringspaces=false,
  basicstyle=\ttfamily,
  emph={label},
  frame=shadowbox, 
  rulesepcolor=\color{blue},
  columns=fixed}

\title{Привеждане в експлоатация на Java приложения}

\author{инж. Божидар ~Бацов}

\institute{Drow Ltd.}

\date{14.12.2010}

\subject{Talks}
% This is only inserted into the PDF information catalog. Can be left
% out. 

\begin{document}

\begin{frame}
  \titlepage
\end{frame}

\begin{frame}{Съдържание}
  \tableofcontents[pausesections]
\end{frame}

\section{Jar файлове}

\begin{frame}{Различни типове приложения}
  \transdissolve
  \begin{itemize}
  \item Десктоп приложения
  \item Приложения, които се изпълняват в браузъра
  \item Уеб приложения(server side)
  \item Enterprise приложения
  \end{itemize}
\end{frame}

\begin{frame}{Различни типове Java архиви}
  \transdissolve
  \begin{itemize}
  \item Десктоп приложения
    \begin{itemize}
      \item jar - java archive
    \end{itemize}
  \item Приложения, които се изпълняват в браузъра
    \begin{itemize}
      \item jar
    \end{itemize}
  \item Уеб приложения(server side)
    \begin{itemize}
      \item war - web archive
      \item нуждаят се от допълнителна програма(web container), в
        контекста на която биват изпълнявани
    \end{itemize}
  \item Enterprise приложения
    \begin{itemize}
      \item ear - enterprise archive
      \item нуждаят се от допълнителна програма(enterprise
        container/application server), в контекста на която биват изпълнявани
    \end{itemize}
  \end{itemize}
\end{frame}

\begin{frame}{Какво представлява един jar файл?}
  \transdissolve
  \begin{itemize}
  \item zip архив на компилираните класове
  \item манифест
  \item ресурси - картинки, файлове с преводи и т.н.
  \item библиотека
  \item изпълним файл(програма)
    \begin{itemize}
      \item java -jar somejar.jar
    \end{itemize}
  \end{itemize}
\end{frame}

\begin{frame}{Подобрена интеграция с операционните системи на Java приложенията}
  \transdissolve
  \begin{itemize}
  \item Solaris
  \item Windows
    \begin{itemize}
      \item launch4j
      \item izpack
    \end{itemize}
  \item OSX
  \item UNIX
  \end{itemize}
\end{frame}


\subsection{Инструментът jar}

\begin{frame}{Инструментът jar}
  \begin{itemize}
  \item Конзолен инструмент за създаване на jar файлове
  \item Моделиран е върху класическия UNIX инструмент tar
  \item Обща употреба
    \begin{itemize}
    \item jar cvf JARFileName File1 File2 . . .
    \item jar cvf CalculatorClasses.jar *.class icon.gif
    \item jar options File1 File2 . . .
    \end{itemize}
  \end{itemize}
\end{frame}

\subsection{Манифест}

\begin{frame}{Манифест}
  \transdissolve
  \begin{itemize}
  \item Специален файл, който описва даден jar
  \item Казва се MANIFEST.MF
  \item Живее в META-INF поддиректорията на архива
  \item Трябва да завършва с празен ред
  \item Създаване на jar с манифест
    \begin{itemize}
    \item jar cfm MyArchive.jar manifest.mf com/mycompany/mypkg/*.class
    \end{itemize}
  \item Обновяване на манифест
    \begin{itemize}
    \item jar ufm MyArchive.jar manifest-additions.mf
    \end{itemize}
  \end{itemize}
\end{frame}

\begin{frame}{Изпълними jar файлове}
  \transdissolve
  \begin{itemize}
  \item Създаване
    \begin{itemize}
    \item jar cvfe MyProgram.jar com.mycompany.mypkg.MainAppClass files to add
    \end{itemize}
  \item Изпълняване
    \begin{itemize}
    \item java -jar MyProgram.jar
    \end{itemize}
  \item Указване на изпълнимия клас през манифеста
    \begin{itemize}
    \item Main-Class: com.mycompany.mypkg.MainAppClass
    \end{itemize}
  \end{itemize}
\end{frame}

\begin{frame}[fragile]
  \frametitle{Запечатване на пакети}
  \transdissolve
\begin{lstlisting}
Manifest-Version: 1.0

Name: com/mycompany/util/
Sealed: true
Name: com/mycompany/misc/
Sealed: false
\end{lstlisting}
\end{frame}

\section{Java Web Start}

\begin{frame}{Java Web Start}
  \transdissolve
  \begin{itemize}
  \item технология за доставяне на приложения посредством браузър
  \item приложенията се изпълняват извън браузъра
  \item приложенията се изпълняват с много ограничени права, ако не са
    цифрово подписани
  \end{itemize}
\end{frame}

\begin{frame}[fragile]
  \frametitle{JNLP}
  \transdissolve
\begin{lstlisting}[basicstyle=\tiny]
<?xml version="1.0" encoding="utf-8"?>
    <jnlp spec="1.0+" codebase="http://localhost:8080/calculator/" href="Calculator.jnlp">
    <information>
    <title>Calculator Application</title>
    <vendor>Euler</vendor>
    <description>A Calculator</description>
    <offline-allowed/>
    </information>
    <resources>
    <j2se version="1.5.0+"/>
    <jar href="Calculator.jar"/>
    </resources>
    <application-desc/>
</jnlp>
\end{lstlisting}
\end{frame}

\section{Applets}

\begin{frame}{Аплети}
  \transdissolve
  \begin{itemize}
  \item Java приложения вградени в HTML уеб страници
  \item Изпълняват се в самия браузър на потребителя
  \item Необходимо е да имате инсталиран Java plug-in за да може да ги използвате
  \end{itemize}
\end{frame}

\begin{frame}[fragile]
  \frametitle{Прост аплет}
  \transdissolve
\begin{lstlisting}[basicstyle=\tiny]
public class NotHelloWorldApplet extends JApplet {
  public void init() {
    EventQueue.invokeLater(new Runnable() {
      public void run() {
        JLabel label = new JLabel("Not a Hello, World applet", SwingConstants.CENTER);
        add(label);
      }
    });
  }
}
\end{lstlisting}
\begin{lstlisting}[basicstyle=\tiny]
<applet code="NotHelloWorldApplet.class" width="300" height="300">
</applet>
\end{lstlisting}

\end{frame}


\section*{Заключение}

\begin{frame}{Заключение}
  \transdissolve
  % Keep the summary *very short*.
  \begin{itemize}
  \item
    Jar файловете са стандартната единица за разпространение на Java код.
  \item
    Java Web Start ви дава възможност да разпространяване лесно
    десктоп приложения написани на Java посредством браузър.
  \item
    Аплетите са прост начин да вграждате Java код в уеб страници.
  \end{itemize}
  
  % The following outlook is optional.
  \vskip0pt plus.5fill
  \begin{itemize}
  \item
    Следващият път:
    \begin{itemize}
    \item
      Обработка на изключения,водене на журнал, очаквания и техники за дебъгване
    \end{itemize}
  \end{itemize}
\end{frame}

\begin{frame}{Въпроси}
  \transdissolve
  \begin{center}
    \LARGEТук е момента да зададете вашите въпроси! :-)
  \end{center}
\end{frame}

\begin{frame}{Край}
  \transdissolve
  \begin{center}
    \LARGEБлагодаря Ви за вниманието!
  \end{center}
\end{frame}

\end{document}

%%% Local Variables: 
%%% mode: latex
%%% TeX-master: t
%%% End: 
