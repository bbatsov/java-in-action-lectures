\documentclass{beamer}

% Copyright 2010 Drow Ltd.
% 
% In principle, this file can be redistributed and/or modified under
% the terms of the GNU Public License, version 2.
% 
% However, this file is supposed to be a template to be modified
% for your own needs. For this reason, if you use this file as a
% template and not specifically distribute it as part of a another
% package/program, I grant the extra permission to freely copy and
% modify this file as you see fit and even to delete this copyright
% notice. 

\mode<presentation>
{
  \usetheme{Warsaw}
  \setbeamercovered{transparent}
}

% \usebackgroundtemplate{
% \includegraphics[width=\paperwidth,
% height=\paperheight]{images/mercatis_background}
% }

\usepackage{times}
\usepackage[utf8]{inputenc}
\usepackage[english,bulgarian]{babel}
\usepackage[T2A]{fontenc}

\usepackage{listings}
\lstset{language=Java,captionpos=b,tabsize=2,frame=lines,keywordstyle=\color{blue},commentstyle=\color{gray},stringstyle=\color{green},numbers=left,numberstyle=\tiny,numbersep=5pt,breaklines=true,showstringspaces=false,basicstyle=\footnotesize,emph={label}}

\title[Въведение в средата за програмиране Java] % (optional, use only with long paper titles)
{Въведение в средата за програмиране Java}

\subtitle
{Преглед на историята и възможностите на Java. Представяне на
  популярни среди за разработка и просто приложения.} % (optional)

\author[инж. Божидар Бацов]{инж. Божидар ~Бацов}

\institute[Drow Ltd.]{Drow Ltd.}

\date[Java Begins]{26.10.2010 / Въведение в средата за програмиране Java}

\subject{Talks}
% This is only inserted into the PDF information catalog. Can be left
% out. 

% \pgfdeclareimage[height=0.5cm]{mercatis-logo}{images/mercatis_logo}
% \logo{\pgfuseimage{mercatis-logo}}

% Delete this, if you do not want the table of contents to pop up at
% the beginning of each subsection:
\AtBeginSubsection[]
{
  \begin{frame}<beamer>{Съдържание}
    \tableofcontents[currentsection,currentsubsection]
  \end{frame}
}


% If you wish to uncover everything in a step-wise fashion, uncomment
% the following command: 

% \beamerdefaultoverlayspecification{<+->}


\begin{document}

\begin{frame}
  \titlepage
\end{frame}

\begin{frame}{Outline}
  \tableofcontents
  % You might wish to add the option [pausesections]
\end{frame}


% Since this a solution template for a generic talk, very little can
% be said about how it should be structured. However, the talk length
% of between 15min and 45min and the theme suggest that you stick to
% the following rules:  

% - Exactly two or three sections (other than the summary).
% - At *most* three subsections per section.
% - Talk about 30s to 2min per frame. So there should be between about
% 15 and 30 frames, all told.

\section{Средата за програмиране Java}

\subsection[Кои сме ние?]{Кои сме ние?}

\begin{frame}{Frame title}
  \begin{itemize}
  \item
    item one
  \item
    item two
  \end{itemize}
\end{frame}

\section*{Заключение}

\begin{frame}{Заключение}

  % Keep the summary *very short*.
  \begin{itemize}
  \item
    Java \alert{е много повече от език за програмиране}.
  \item
    JVM \alert{е целевата среда за изпълнение} на Java приложенията, а
    не физическата процесорна микроархитектура.
  \item
    За пълноценна работа с езикът и платформата Java човек трябва да
    се запознае с доста инструменти.
  \end{itemize}
  
  % The following outlook is optional.
  \vskip0pt plus.5fill
  \begin{itemize}
  \item
    Следващият път:
    \begin{itemize}
    \item
      Основния положения в езикът Java
    \item
      Повече примери, по-малко общи приказки
    \end{itemize}
  \end{itemize}
\end{frame}


\begin{frame}{Въпроси}
  \begin{center}\LARGEТук е момента да зададете вашите въпроси! :-)\end{center}
\end{frame}


\begin{frame}{Край}

  \begin{center}
    \LARGEБлагодаря Ви за вниманието!
  \end{center}
  
\end{frame}



\end{document}



%%% Local Variables: 
%%% mode: latex
%%% TeX-master: t
%%% End: 
